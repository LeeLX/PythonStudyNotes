\documentclass[a4paper,12pt]{article}
\usepackage{ctex}
\usepackage{color}
\usepackage{amsmath}
\usepackage{graphicx}
\usepackage{pythonhighlight}
\usepackage[top=2.5truecm,bottom=2.5truecm,left=2.3truecm,right=2.3truecm,includefoot,xetex]{geometry}
\usepackage{fancyhdr}
\usepackage{subfigure}


%字体设置
\setCJKmainfont[BoldFont=SimHei]{SimSun}
\setCJKfamilyfont{hei}{SimHei}
\setCJKfamilyfont{kai}{KaiTi}
\setCJKfamilyfont{fang}{FangSong}
\newcommand{\hei}{\CJKfamily{hei}}
\newcommand{\kai}{\CJKfamily{kai}}
\newcommand{\fang}{\CJKfamily{fang}}

%字号设置
\newcommand{\chuhao}{\fontsize{42pt}{\baselineskip}\selectfont}
\newcommand{\xiaochuhao}{\fontsize{36pt}{\baselineskip}\selectfont}
\newcommand{\yihao}{\fontsize{28pt}{\baselineskip}\selectfont}
\newcommand{\erhao}{\fontsize{21pt}{\baselineskip}\selectfont}
\newcommand{\xiaoerhao}{\fontsize{18pt}{\baselineskip}\selectfont}
\newcommand{\sanhao}{\fontsize{15.75pt}{\baselineskip}\selectfont}
\newcommand{\xiaosanhao}{\fontsize{15pt}{\baselineskip}\selectfont}
\newcommand{\sihao}{\fontsize{14pt}{\baselineskip}\selectfont}
\newcommand{\xiaosihao}{\fontsize{12pt}{\baselineskip}\selectfont}
\newcommand{\wuhao}{\fontsize{10.5pt}{\baselineskip}\selectfont}
\newcommand{\xiaowuhao}{\fontsize{9pt}{\baselineskip}\selectfont}
\newcommand{\liuhao}{\fontsize{7.875pt}{\baselineskip}\selectfont}
\newcommand{\qihao}{\fontsize{5.25pt}{\baselineskip}\selectfont}
%设置 section 属性
\makeatletter
\renewcommand\section{\@startsection{section}{1}{\z@}%
{-1.5ex \@plus -.5ex \@minus -.2ex}%
{.5ex \@plus .1ex}%
{\normalfont\xiaosanhao\CJKfamily{hei}}}
\makeatother
%设置 subsection 属性
\makeatletter
\renewcommand\subsection{\@startsection{subsection}{1}{\z@}%
{-1.25ex \@plus -.5ex \@minus -.2ex}%
{.4ex \@plus .1ex}%
{\normalfont\sihao\CJKfamily{hei}}}
\makeatother
%设置 subsubsection 属性
\makeatletter
\renewcommand\subsubsection{\@startsection{subsubsection}{1}{\z@}%
{-1ex \@plus -.5ex \@minus -.2ex}%
{.3ex \@plus .1ex}%
{\normalfont\xiaosihao\CJKfamily{hei}}}
\makeatother



%行间距离
\linespread{1.5}
\setlength{\baselineskip}{20pt}

%重定义
\renewcommand{\contentsname}{\centerline { \xiaoerhao \textbf{目 \quad 录}}}
\renewcommand{\abstractname}{摘要}
\renewcommand{\refname}{参考文献}
\renewcommand{\figurename}{图}
\renewcommand{\tablename}{表}

%\newcommand{\HRule}{\rule}{\linewidth}{0.5mm}}
\newcommand{\HRule}{\rule{\linewidth}{0.5mm}}
\newcommand{\Hrule}{\rule{\linewidth}{0.3mm}}

\begin{document}
\begin{center}
\HRule \\
\vspace{6mm} %magenta \color{Mahogany}
{ \erhao \kai {Python基本语法}}
\HRule \\
\end{center}

\tableofcontents

\newpage

\section{序列}
\subsection{序列定义及分类}
Python的基本数据结构,序列中的元素有序排列,每个元素被分配一个序号(元素的位置,即索引index)。通过指定一个{\color{red}下标}索引值(a[0]、a[1],索引),可以访问到序列中的任何一个元素;通过{\color{red}切片}方式一次可以得到多个元素(a[0:3],切片)。

序列索引从0开始到N--1结束(N是序列中元素的个数),或者从最后一个元素--1开始到--N结束。


序列分类:{\color{red}字符串、列表、元组}

成员关系运算符:{\color{red}in 、 not in},用于判断一个元素是否属于一个序列。

\subsection{序列操作符}

\subsubsection{索引}
\begin{itemize}
\item sequence[index]
\end{itemize}

{\kai 索引值可以为负值,负数表示从右往左开始计数,最后一个元素索引为--1,倒数第二个为--2,以此类推。}

\subsubsection{切片}
\begin{itemize}
\item sequence[start\_index : end\_index]
\item sequence[start\_index : end\_index : step]
\end{itemize}

获取从开始索引到结束索引之间的一组连续的元素(不包括结束索引对应的元素)。

{\kai 如果没有特别指定开始和结束索引值,切片操作从序列的最开始处开始,直到序列的最末端结束( sequence[ : ] ),且默认步长为1。}

\begin{python}
seq[index]						#获得下标为index的元素(索引)
seq[index1 : index2] 	#获得下标从index1到index2间的元素集合(切片)
seq * num							#序列重复num次
seq1 + seq2 					#连接序列seq1和seq2
element in seq				#判断element元素是否包含在seq中
element not in seq		#判断element元素是否不包含再seq中
\end{python}


\begin{python}
a = "Hello"
a[0]						#"H"
a[4]						#"o"
a[-1]						#"o"
a[-5]						#"H"
a[1:3]					#"el",不包含a[3]
a[-1:-5:-1]			#"olle",不包含a[-5]
a * 2						#"HelloHello"
b = "World."			
c = a + b 			#"HelloWorld."
"H" in a 				#True

\end{python}

\section{列表}
字符串是一种不可变序列类型,只能由字符组成;列表内容可以修改,可以包含任意数据类型,列表中的元素也可以是列表({\kai \color{red}二维列表})。

\subsection{创建列表}
创建列表可以使用方括号( [ ] ),列表元素之间用逗号隔开,或者使用函数list()。
\begin{python}
listName = [elem1, elem2, elem3, ..., elemN]
listName = list(sequence)
\end{python}

\begin{python}
#使用方括号 [] 创建
numList = [1, 2, 3, 4, 5]
strList = ["A", "B", "C", "D"]
mixList = [1, 1.1, "ABC", True]

#使用函数 list() 创建
emptyList = list()								#空列表
numList = list(range(5))
\end{python}

\newpage
\subsection{访问列表}
列表与字符串类似,也可以通过索引和切片两种操作进行访问。
\begin{itemize}
\item 索引
\item 切片
\end{itemize}

\noindent 一维列表遍历
\begin{python}
friends = ["Tom", "Jerry", "Bat"]

#1. 
for friend in friends:
	print("Happy new year,", friend)
	
#2. 
for i in range(len(friends)):
	friend = friends[i]
	print("Happ new year,", friend)
\end{python}

{\noindent \kai  二维列表的访问方式:\color{red}$listName[index1][index2]$}

\begin{python}
#二维列表
list1 = [1001, 1002, 1003]
list2 = [1004, 1005, 1006]
twoDimList = [list1, list2]			#二维列表由list1和list2组成
print(twoDimList[0][0])					#1001
print(twoDimList[0][2])					#1003
print(twoDimList[1][1])					#1005

#遍历二维列表
for i in range(len(twoDimList)):
	for j in range(len(twoDimList[i])):
		print("twoDimList[" + str(i) + "][" + str(j) + "]=", twoDimList[i][j])
		#print("twoDimList[%d][%d]=%d" % (i, j, twoDimList[i][j]))
\end{python}

\subsection{修改列表}
\begin{itemize}
\item 添加元素
\item 修改元素
\item 删除元素
\end{itemize}

\begin{python}
listName.append(elem)					#向列表尾部添加元素elem
listName.insert(index, elem)	#将元素elem插入到列表中的指定索引位置index
listName[index] = newElem			#直接使用赋值符号修改该索引的值
listName.remove(elem)					#删除列表中指定的元素elem,只删除第一个匹配元素
del listName[index]						#删除列表中指定索引位置的元素
\end{python}

\subsection{list常用内建函数}
\begin{python}
cmp(list1, list2)					#比较两个列表中的元素,(>:1; <:-1; =:0)
len(list)									#返回列表元素个数
max(list)									#返回最大元素
min(list)									#返回最小元素
sorted(list)							#返回一个从小到大排序的列表
sum(list)									#列表求和(元素为数字)
\end{python}

\section{元组}
\subsection{创建元组}
创建元组可以使用圆括号( () ),元组元素之间用逗号隔开,或者使用函数tuple()。元组中的元素也可以是元组({\kai \color{red}二维元组})。

{\kai\sanhao \color{red}元组不可被修改!}

\begin{python}
tupleName = (elem1, elem2, elem3, ..., elemN)
tupleName = tuple(sequence)
\end{python}

\begin{python}
#使用圆括号 () 创建
numTuple = (1, 2, 3, 4, 5)
strTuple = ("A", "B", "C", "D")
mixTuple = (1, 1.1, "ABC", True)

#使用函数 tuple() 创建
numTuple = tuple(range(5))
\end{python}
\subsection{访问元组}
元组访问与列表访问类似。

\subsection{tuple常用函数}
\begin{python}
cmp(tuple1, tuple2)					
len(tuple)									
max(tuple)									
min(tuple)									
sorted(tuple)							
sum(tuple)									
\end{python}

\section{字典--映射类型}

映射类型也是Python中一种常见的数据结构类型,这种类型由键(Key)值(Value)对组成。一个键只能对应一个值,但是多个键可以对应相同的值。

\subsection{创建字典}

创建字典可以使用花括号( \{ \} ),每一个“键值对”元素用逗号分隔,“键值对”中间用冒号分隔。

{ \kai ** 还可以使用函数dict()或函数fromkeys()}

\begin{python}
#使用花括号 {} 创建(普通方式)
dictName = {key1:value1, key2:value2, key3:value3, ..., keyN:valueN}

#使用 dict() 创建
a = dict()												#空字典
# *** (key,value)形式的列表
dictName = dict( [ (key1,value1), (key2,value2) ] )
#(key,value)形式的元组
dictName = dict( ( (key1,value1), (key2,value2) ) )

# *** 使用 {}.fromkeys() 创建
#指定value值
dictName = {}.fromkeys( (key1, key2, key3), value )
#不指定value值,默认为None
dictName = {}.fromkeys( (key1, key2, key3) )
\end{python}

\subsection{访问字典}
使用方括号,指定相应的键(类似索引)。
\begin{python}
dictName[key]
\end{python}

{\kai \color{red} 字典中的值是无序的 },因此遍历字典的方式与遍历序列不同。

\begin{python}
#1. 直接迭代
for key in dictName:
	print(dictName[key])


#2. 使用 keys() 函数
for key in dictName.keys():		#keys()函数返回dict_keys类型
	print(dictName[key])

#3. 使用 items() 函数
for (key, value) in dictName.items():
	print(dictName[key])
\end{python}

\begin{python}
studentDict = {1001:"A", 1002:"B", 1003:"C"}
for (key, value) in studentDict.items():
	print("studentDict[%s] = %s" % ( str(key), str(value) ))
\end{python}

\subsection{更新字典}
\subsubsection{修改、添加}
通过赋值语句进行字典元素的修改与添加。
\begin{python}
dictName[key] = newValue
\end{python}
{\kai 如果key在原字典中不存在,则直接将新元素(键值对)添加到字典中;如果key已经存在,则newValue会覆盖原来字典key对应的值,从而实现了字典元素的修改功能,同时也相当于删除了原来的key对应的元素。}

\subsubsection{删除元素}
删除字典元素可以使用:del()函数、pop()函数或者del语句。

\begin{python}
#1. del函数
del(dictName[key])				#删除键为key的元素

#2. pop()函数
dictName.pop(key)					#删除键为key的元素,返回value值

#3. del 语句
del dictName[key]					#删除键为key的元素
\end{python}



\section{集合}
无序不重复的元素集;集合分为可变集合和不可变集合。

\subsection{创建集合}
创建可变集合可以使用花括号( \{ \} ),每一个元素使用逗号进行分隔;或者使用set()函数。

创建不可变集合,使用frozenset()。

\begin{python}
#使用普通方式创建
setName = {elem1, elem2, elem3, ..., elemN}

#使用函数 set() 创建
setName = set( [ "A", "B", "C", "D" ] )

#使用函数 frozenset() 创建
frozenSetName = frozenset( [ "A", "B", "C", "D" ] )

copyOfSetA = set(a)							#a可以是string,list,tuple,set,dict(keys)
copyOfFrozenSetA = frozenset(a)
\end{python}

\newpage
\subsection{访问集合}
集合中的元素无序并且不重复,无法获取集合中某个指定的元素,只能遍历整个集合来访问其中的所有元素。

\begin{python}
for elem in setName:
	print(elem)

#集合遍历例子	
numSet = {1,2,3,4,5}
for tmpNum in numSet:
	print(tmpNum)
\end{python}

\subsection{更新集合}
\subsubsection{添加元素}
\begin{itemize}
\item add()、update()
\end{itemize}

\begin{python}
setA.add(elem)				#添加一个元素elem
setB.update(setA)			#将集合setA中所有元素添加到集合setB中,去掉重复元素
\end{python}

\subsubsection{删除元素}

\begin{itemize}
\item remove()、pop()、discard()、clear()
\end{itemize}

\begin{python}
# remove()
setName.remove(elem)				#删除元素elem

# pop()
setName.pop()						#删除集合中的第一个元素,并返回该元素

# discard()
setName.discard(elem)			#删除元素elem

# clear()
setName.clear()					#删除集合中的所有元素
\end{python}



\end{document}
